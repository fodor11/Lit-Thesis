\Chapter{A tűz fizikája}

% TODO: Leírni, hogy a tűz fizikája (beleértve a megjelenítés optikáját) hogyan működik nagyvonalakban!
Az ősi görögök és alkemisták úgy vélték, hogy a tűz maga is egy elem a föld, levegő és víz mellett. Egy elemet napjainkban azonban az alapján definiálnak, hogy a tiszta anyagban mennyi proton található. A tűz viszont számos más anyagból áll, így nem lehet elem. 

A tűz nagyrészt forró gázok keveréke. A lángok egy kémiai reakció - elsősorban a levegőben lévő oxigén és valamilyen ``üzemanyag'' között - következményei. Ilyen üzemanyag lehet a fa vagy a propán is.

Más termékek mellett a reakció széndioxidot, gőzt, fényt és hőt termel. Ha a láng elég forró, a gázok ionizálódnak és egy új halmazállapotba lépnek: plazmává válnak. Fém égetése, mint például a magnéziumé, képes ionizálni az atomokat és plazmát formázni. Az oxidáció ezen formája a forrása a plazma fáklya intenzív fényének és hőjének.

Egy átlagos tűzben is lezajlik némi ionizáció, ám a lángban található anyagok nagyrésze gáz formában van jelen, így a tűz halmazállapotát leginkább a gáz jellemzi, egy kevés plazmával vegyítve.

A láng struktúrája annak függvényében változik, hogy melyik részét vizsgáljuk. A láng alapjánál oxigén és üzemanyaggőz keveredik elégetlen gázokként. Ennek a résznek a tartalma az égő üzemanyagtól függ. Efelett a régió felett lépnek egymással reakcióba a molekulák az égés során. 

Ezen régió felett az égés már lezajlott, így itt a kémiai reakció végtermékei találhatóak. Ez legtöbb esetben szén-dioxid és vízgőz. Ha az égés nem tökéletes, akkor még pernye és hamu és fellelhető mellettük. Tökéletlen égés következtében további gázok is keletkezhetnek, főleg olyan ``piszkos'' tüzelőanyagok esetén, mint a szén-monoxid és a kén-dioxid.

Habár nem könnyű észrevenni, de lángok más gázokhoz hasonlóan tágulnak. Ezt részben azért nehéz észrevenni, mert a lángnak csak azon részeit látjuk, melyek elég forróak ahhoz, hogy fényt bocsátsanak ki. A lángok azért nem gömb alakúak, mert a környező levegőnél kevésbé sűrűbbek, így felszállnak. 

A lángok színe a hőmérsékletüket és az üzemanyag kémiai összetételét jelzi. A láng izzó fényt bocsát ki, ahol a kék szín a legnagyobb energiájú fényt jelzi (ahol a legforróbb a láng), a kisebb energiájú fény (ahol a láng hűvösebb) pedig vörösebb.
\cite{firePhysics1, firePhysics2}

\section{A tűz természetéből fakadó nehézségek}
Az égést, mint kémiai jelenséget, még ha ismernénk is minden részletet sem tudnánk elfogadható időn belül szimulálni  \cite{firestarter}.

A tűz a jelenetek egyik drámai eleme, így azt a lehető legnagyobb szinten kell tudni befolyásolni amellett, hogy közben hiteles is maradjon. Kiszámíthatatlannak és összetettnek kell tűnnie, ugyanakkor az adott körülményeknek megfelelően felismerhető struktúrájúnak kell lennie. Ebből a bonyolultságból és a szemlélői elvárásokból kifolyólag nem túl életképes a tűz animálása közvetlen numerikus szimuláció segítségével, ugyanis:
\begin{itemize}
\item
A numerikus szimulációk nagyban függenek a részletesség szintjétől. Ahogy egy 3D-s szimuláció felbontása finomodik, a számítási összetettsége legalább $O(n^3)$ szerint nő. Ha akár csak egy kisebb tüzet is szeretnénk részleteiben megjeleníteni, az nagyon magas számítási igényű rendereléshez vezet.
\item
Azok a (nagy számú) faktorok, melyek a tűz megjelenését határozzák meg különféle körülmények között, egy egymástól függő paraméterekből álló teret alkotnak. Ilyen fizikai rendszerre nehéz olyan függvényeket írni, melyek segítségével könnyen lehetne befolyásolni a tüzet. 
\item
A tűz maga a zűrzavar. Apró változtatások a kiinduló környezetben rendkívül eltérő eredményeket produkálnak. Az animálás szemszögéből nehéz egy kívánt eredmény felé terelni a folyamatokat \cite{ArNiStructuralModeling}.
\end{itemize}

Más-más alkalmazási területek különféle szempontokat tartanak szem előtt, mikor a tűz természetéről, megjelenéséről, és a megjelenítési időről van szó. Például:

Egy videojátékban az elsődleges szempont hogy valós időben (25-60 frame per secundum) lehessen renderelni a jelenséget, a kinézetnek is a lehetőségek szerint valóságosnak kell tűnnie. A legtöbb esetben csupán díszítő funkciót tölt be, így a terjedés fizikája és az interakció kevésbé lényeges.

Egy mozifilmben a legfontosabb, hogy minél élethűbb legyen, és esetlegesen lehessen kontrollálni a kinézetét. A renderelésnek itt már nem kell valós idejűnek lennie, illetve sem a tűz terjedésének fizikája, sem a vele való interakció nem játszik szerepet.

Egy virtuális valóságban - például egy tűzoltó szimulációs szoftverben - ismét szempont a valós idejű renderelés, de itt a tűzzel való interakció lehetősége és a terjedésének fizikája egyaránt fontosabb, mint maga a megjelenés.