\Chapter{Saját tűz implementációk}

% TODO: Ide kellene sorban bemutatni az implementációkat!

\section{Fejlesztői környezet}

\subsection{OpenGL}
Az OpenGL egy platform- és nyelvfüggetlen alkalmazásprogramozási felület (API) 2 és 3D-s vektorgrafikus megjelenítéshez. Az API használatán keresztül elérhetjük a grafikus kártyát, így hardveresen gyorsított megjelenítést valósíthatunk meg a számítógépen. A Silicon Graphics Inc. kezdte fejleszteni, majd 1992-ben adták ki először. Széles körben alkalmazzák az iparban, többek között számítógéppel segített tervezésben (CAD), virtuális valóságokban, tudományos vizualizációk során, információ szemléltetésben, repülő szimulációkban és végül, de nem utolsó sorban a számítógépes játékokban. 2006 óta a nonprofit Khronos csoport vette át a fejlesztését. \cite{wikiOGL}

\subsection{Freeglut}
Az OpenGL Utility Toolkit (GLUT) egy olyan platformfüggetlen könyvtár, amely gondoskodik a rendszerspecifikus feladatokról, melyekre ablak kezelés, OpenGL kontextusok inícializálása és bemeneti események (billentyűzet, egér) kezelése esetén van szükség. Használata gyorsan elterjedt, ugyanis egyszerű, széles körben elérhető és hordozható volt. Azonban 1998 augusztusa óta nem fejlesztették tovább, így a kód elöregedett. A FreeGLUT egy ingyenes, nyílt forráskódú alternatíva ehhez. Továbbra is fejlesztés alatt áll, és a GLUT-hoz képest több funkcióval, és kevesebb hibával rendelkezik.

\subsection{OpenGL Extension Wrangler Library (GLEW)}
A GLEW egy platformfüggetlen C/C++ könyvtár, amely segíti az OpenGL bővítmények (extensions) lekérdezését és betöltését. Futásidőben megállapítja, hogy az adott célhardveren mely OpenGL bővítmények támogatottak. Az összes bővítményt kigyűjti egyetlen, a számítógép által generált header fájlba, mely a hivatalos bővítmény listából készül. Több operációs rendszeren is tesztelték, többek között Windows-on, Linux-on, Mac OS X-en, FreeBSD-n, Irix-en és Solaris-on. A 2.1.0 verzió már az OpenGL 4.6-os verzióját is támogatja. 

\subsection{OpenGL Mathematics}
Az OpenGL Mathematics (GLM) egy OpenGL Shading Language (GLSL) specifikációkon alapuló, csak header fájlokból álló C++ könyvtár a grafikus szoftverekhez. Olyan osztályokkal és függvényekkel szolgál, melyek elnevezési konvenciói és funkciói megegyeznek a GLSL-ben találhatókkal, íg aki a GLSL-t ismeri, az tudja használni a GLM-et is. Ezen felül azonban olyan bővítmény rendszerrel is bír, mely szintén megfelel a GLSL bővítmény konvencióknak, és ez további képességekkel szolgál: mátrix transzformációk, random számok, zaj, stb.

\subsection{Simple OpenGL Image Library}
A Simple OpenGL Image Library (SOIL) egy platformfüggetlen C könyvtár, melyet elsősorban textúrák betöltésére használhatunk. Elvégzi a szükséges lépéseket, melyeket az OpenGL igényel textúrák előkészítésénél. A könyvtár segítségével nem csak betölteni, hanem menteni is lehet képeket. Mivel kis méretű, első sorban statikus könyvtárként való használatra szánták. Tesztelték Windows-on, Linux-on és Mac-en. 

\section{A kiinduló projekt}

% domborzat
Az következő implementációkat egy már elkészült, korábbi projekt keretein belül jelenítem meg. GLEW 2.0-t használtam, tehát az OpenGL 4.5-ös verziójának funkciói álltak rendelkezésemre. 

A projekt egy általam készített domborzatot tölt be, melyhez az adatokat különböző képfájlokból olvassa be. A magasságértékeket egy szürkeárnyalatos kép képpontjaiban tároltam, a hozzájuk tartozó nedvesség értékeket pedig egy másik, szintén szürkeárnyalatos képfájlban. A színeket egy olyan képből olvastam ki, ahol az egyik tengely a magasságot, a másik pedig a nedvességet jelölte, így az adott csúcspont színe a magasság és nedvesség értékéből kapható meg. A domborzathoz 260100 darab csúcspontot használtam fel. Az égbolt és a hold is gömb alakú, melyekhez a csúcspontokat, textúra koordinátákat, normál vektorokat és az ezekhez szükséges indexeket is az általam készített Sphere osztály generálja. Az ég 4225, a hold 1089 csócspontból áll. Mindezek megjelnítéséért és aktualizálásáért az Environment osztály felel.

A program így 430 és 460 fps (frame per secundum) között fut, azaz másodpercenként nagyjából ennyiszer rajzolja ki a jelenetet.

\subsection{Textúrák}
A textúrák betöltéséért, paraméterezéséért, illetve az egyéb információkkal szolgáló képek beolvasásáért a TextureLoader osztályom felel a SOIL könyvtár segítségével. 

Textúrát betölthetünk egyszerűen a SOIL\_load\_OGL\_texture() függvényével is , ebben az estben csupán a textúra paramétereket kell beállítani, azaz hogy a kép nyújtása vagy zsugorítása esetén milyen módszerrel mintavételezze a textúrát, illetve a $[0, 1]$ intervallumot meghaladó textúra koordináták esetén milyen módszerrel töltse ki az érintett részeket. Ennek a hátránya azonban, hogy a mintavételezési lehetőségek között nincs olyan, amely a zsugorított textúrák esetén szép képet eredményezne. Ezért készítettem a loadMipMappedTexture() függvényt, amely mipmap-et is generál a betöltött képhez a glGenerateMipmap() függvény segítségével. Mivel itt csak a SOIL\_load\_image() függvényt használom a kép beolvasásához, a textúra betöltéséről, azonosító generálásról és a pixel tárolási módról is gondoskodni kellett. Előnye viszont, hogy az adott kép több, kisebb méretben is legenerálódik, így a zsugorított textúrák is sokkal szebben mutatnak.

A loadImage() függvény segítségével lehet egyszerűen képeket beolvasni. Ehhez társul a getPixelColor() függvény is, mely visszaadja a koordinátái segítségével megadott pixel RGB színét. Az értékeket a $[0, 1]$ intervallumra képeztem le ($\frac{\text{pixel érték}}{255}$).

\subsection{Shader-ek}
A shader-ek betöltésére és használatára egy külső könyvtárat használtam, melynek funkciói a ``tdogl'' névteren belül találhatóak, innen lehet felismerni őket. Ez elvégzi a szükséges OpenGL beállításokat, és nagyban leegyszerűsíti az árnyaló programok használatát. A shader-ek Phong-féle megvilágítási modellt implementálnak, ezen belül is kétféle változatot készítettem. Az egyikben csúcspontonként megadhatók különböző színek, ezt használtam a domborzat kirajzolásához. A másikban pedig egy objektum kirajzolásához egy szín adható meg uniform-ként, tehát így minden csúcspont színe egyforma. Utóbbit azon objektumokra alkalmaztam, melyek színét leginkább a textúra definiálja (az égbolt és a hold).

\subsection{Camera osztály}
% camera: transzformációk, ütközésvizsgálat, időmérés
A nézeti és projekciós mátrixok transzformációit a Camera osztályban tartom számon. Ez az osztály felel a térben való mozgásért, illetve az Y (jobbra/balra fordulás) és az X (fel/le nézés) tengely mentén való forgatásért. 
Az elfordulási szögek radiánban vannak számontartva, és a túlcsordulások elkerülése érdekében az Y tengely menti fordulást a $[-\pi \cdot 2, \pi \cdot 2]$, az X tengely mentit pedig $[-\frac{\pi}{2}, \frac{\pi}{2}]$ intervallumra szűkítem le. Nem tartom számon külön a nézeti, és külön a projekciós mátrixot, hanem a kettő szorzatából eredő ``camera'' mátrixot használom. Az updateCamera() metódust minden kirajzolás elején meg kell hívni, ugyanis ez számolja ki a pillanatnyi camera mátrixot, és állítja be minden shader-ben uniform-ként. 

Egyszerű ütközésvizsgálatra is ebben az osztályban van lehetőség, ehhez meg kell adni az akadályok pozícióinak listáját, illetve egy akadály sugarát, melyet az alapjának a befoglaló köre határoz meg. Az ütközésvizsgálat csak kétdimenziós, az XZ síkon történik. Eredetileg fákkal történő ütközésvizsgálathoz lett írva, de a tűzön való áthaladást is könnyedén meg lehet vele akadályozni, ha szükséges. A kamera új pozíciójának kiszámítása előtt letárolom a pillanatnyi pozíciót, és amennyiben az új pozícióban ütközés történik, az előző pozíció értéke marad meg. Az ütközést rendkívül egyszerűen állapítom meg: a kamera (X, Z) pozíciója és az akadály (X, Z) pozíciója közti távolságot számolom ki minden akadályra. Amennyiben ez a távolság kisebb, mint a megadott sugár, ütközés történt. Nem golyóálló a módszer, ellenben rendkívül gyors.

Az új pozíció meghatározásához számon kell tartani az előző kirajzolás óta eltelt időt, hiszen ha minden kirajzolás esetén ugyanannyival mozdulna el a kamera az adott irányba, akkor a hardver gyorsaságától is függne a sebessége. Mivel a glutGet(GLUT\_ELAPSED\_TIME) felbontása nem túl nagy a mai hardverek sebességéhez képest, előfordulhat, hogy többször is nullát kapunk az eltelt idő számításakor, így a kamera nem mozdul. Helyette a Windows QueryPerformanceCounter() függvényét használtam, mely megbízhatóbb e téren.

\subsection{Frame per secundum mérés}
% fps mérő
Annak érdekében, hogy a kirajzolt jelenetek számításigényéről információt szerezzünk a későbbiekben, szükséges volt egy fps mérő implementálása is. Ezt két függvény segítségével valósítottam meg: calcFps() és printFps(). Előbbi kirajzolásonként kiszámítja az aktuális fps-t, amelyet az $\frac{1}{\text{eltelt idő (sec)}}$ képletből számoltam. Utóbbi a refreshTime változóban (ms) meghatározott időközönként kiírja az arra az intervallumra vonatkoztatott átlag fps-t, amelyet az fps-ek összege és a kirajzolt képek számának a hányadosa ad meg. A képernyőre való kiíratáshoz a glutBitmapCharacter(...) függvényt használtam, amely bitmap karaktereket ír ki a képernyőre. 


\section{Kétdimenziós sprite, illetve billboard tűz}

% osztályok

% VBO-k + textúra

% shaderek

% átlátszóság


\section{Háromdimenziós részecskerendszer alapú tűz}
















