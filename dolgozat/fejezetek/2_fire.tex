\Chapter{A tűz megjelenítése}

% TODO: Összeszedni az eddigi eredményeket a tűz megjelenítésével kapcsolatban!

\section{Megvalósítás a fontosabb játék motorokban}

A háromdimenziós játékok felemelkedése az 1990-es években kezdődött. Az első játékmotorok az id Software feljelszőinek köszönhetőek. Ahelyett, hogy minden játékot egyenként a semmiből építettek volna fel, más fejlesztők által készített modulokat használtak fel. Megtervezték a saját grafikájukat, karaktereiket, fegyvereiket és pályáikat, melyek lényegében a játék tartalmát nyújtották. Így már szét lehetett választani a játékspecifikus szabályokat és adatokat olyan alap koncepcióktól, mint például az ütközésvizsgálat. Ennek köszönhetően a fejlesztői csapatokat lehetett bővíteni, hiszen egyszerre több modulon is lehetett párhuzamosan, egymástól függetlenül dolgozni. Ez lehetővé tette továbbá, hogy a fejlesztők külön területekre specializálódjanak.
\cite{wikiGameEngine}

A korábbi, kétdimenziós játékokban természetesen csak sprite (kétdimenziós bitmap) alapú tüzekkel találkozhatunk. Az id tech 1 motor sem kivétel, habár hármodimenziósnak tűnnek a vele készült játékok (pl.: Doom I, Doom II), a benne lévő objektumok szintén sprite-ok, melyek mindig a néző felé fordulnak. Az ilyen játékmotorokat 2.5 dimenziós motoroknak is nevezhetjük. 


%Sprite tűz:
%1993 id tech 1/doom engine https://en.wikipedia.org/wiki/Doom_engine +  https://doomwiki.org/wiki/Doom_rendering_engine 
% Doom II: Hell on Earth (id tech 1 ez is)

\subsection{Quake engine 1996}
% MDL format: http://fabiensanglard.net/quakeSource/quakeSourceRendition.php
% torch animation http://media.moddb.com/images/members/1/240/239733/qv1.gif
A Quake motor 1996-ban készült el a Quake számítógépes játékot meghajtva. Az id tech 1 motorral ellentétben ez már igaz háromdimenziós motor, azaz háromdimenziós adatokat felhasználva állítja elő a kétdimenziós képet. A megjelenítés valós időben történik. A játékban egyaránt használtak fénytérképeket (lightmap) és 3D-s fényforrásokat is. Előbbi segítségével a statikus objektumok fényét lehetett az objektumokra égetni, ezzel rövidítve a kirajzolási időt, utóbbi pedig a dinamikus objektumok fényét adta. \cite{wikiQuake} A játékban a fáklyák tüzét vizsgáltam.\\
A fáklyák kaphattak pislákoló fényt, ám ez rendkívül lassította a kirajzolást, így többnyire csak olyan helyen lehetett alkalmazni ahol kevés felületet világított meg. A tűz maga váltakozó modellek sorozatából állt, melyet MDL fájformátumban tároltak le.\\
 Ez a formátum lehetővé teszi a képkockáról képkockára való animáció tárolását. Egy .mdl fájl jellemzői: 
\begin{itemize} 
\item A modell geimetriai adatai (háromszögekben)
\item Textúra adatok
\item Képkockáról képkockára való animáció
\end{itemize} 
 Minden képkockához külön-külön hozzá lehet rendelni egyebek mellett, hogy mely csúcsok szerepelnek benne. A forműtum lehetővé teszi egy befoglaló gömb megadását is, mely ütközésvizsgálatok esetén lehet fontos, ám ezt nem kötelező megadni benne.  \cite{MDLformat} \\
Tehát a tűznek semmiféle külön fizikai modellezéséről sincs szó (legalábbis valós időben), előre meg van határozva a mozgása képkockákra lebontva. Mint az a mellékelt ábrán is látszik, a tűz lángjain nem lehet átlátni, illetve füstképződés sincs. A játék OpenGL-es portjában a gyorsítás érdekében a modelleket DisplayList-ekbe töltötték be (gl mesh.c).



% 1997 Quake II engine = id tech 2
% 1998 Unreal engine 1
% 1999 id tech 3 engine 
% 2002 Unreal engine 2
% 2004 id tech 4 engine 
% 2004 CryEngine 1 https://github.com/AFCStudio/CRYENGINE-1
% 2007 Unreal engine 3
% 2012 Unreal engine 4 
% 2017 CRYENGINE 5.4.0 https://github.com/CRYTEK/CRYENGINE


% Not open source:
% 2012 Gamebryo 4.0  https://en.wikipedia.org/wiki/Gamebryo
% RenderWare https://en.wikipedia.org/wiki/RenderWare
% Fork_Particle https://en.wikipedia.org/wiki/Fork_Particle
% Source https://en.wikipedia.org/wiki/Source_(game_engine)
% Creation Engine https://en.wikipedia.org/wiki/Creation_Engine
% Crystal Tools https://en.wikipedia.org/wiki/Crystal_Tools
% Enigma Engine https://en.wikipedia.org/wiki/Enigma_Engine
% Frostbite https://en.wikipedia.org/wiki/Frostbite_(game_engine)
% GoldSrc https://en.wikipedia.org/wiki/GoldSrc
% IW engine https://en.wikipedia.org/wiki/IW_engine
% Jade https://en.wikipedia.org/wiki/Jade_(game_engine)
% RenderWare https://en.wikipedia.org/wiki/RenderWare
% Riot engine https://en.wikipedia.org/wiki/Riot_Engine
% RAGE https://en.wikipedia.org/wiki/Rockstar_Advanced_Game_Engine
% Vision https://en.wikipedia.org/wiki/Vision_(game_engine)#Games_using_Vision_Engine
% 2011 id tech 5 
% 2016 id tech 666

