\documentclass[12pt,a4paper]{report}

\usepackage{dolgozat}

\usepackage{hyperref}

\usepackage{listings}
\usepackage{python}
\usepackage{cpp}

\usepackage{wrapfig}

\linespread{1.2}

\begin{document}

\section*{Összegzés}


%feladat
A dolgozat fő célja a tűz és a hozzá kapcsolódó jelenségek megvalósításának áttekintése volt a számítógépi grafikában. Az elérhető játékmotorok vizsgálatán keresztül próbáltam betekintést nyerni a tűz megjelenítéssel kapcsolatos problémáinak megoldásaira, illetve ezeket vizsgáltam valószerűség és számításigény tekintetében. Elsősorban a valós idejű megjelenítéssel foglalkoztam, de kitértem néhány valós időben nem feldogozható módszerre is. Egyes módszereket magam is implementáltam.

%eredmények
% néhány motor áttekintve, történeti ismeretek, praktikák megismerése
%tűz fizikájának, folyamatainak megismerése
% különböző módszerek és lehetőségek feltárása, ötletelés szintjén
% billboard tűz, particle system, post processing implementáció
A legkorábbi játékmotorok vizsgálatával kezdtem, majd időrendben haladtam a modernebb, számításigényesebb megvalósítások irányába. A kutatás során a motorokkal kapcsolatban történeti ismeretekre is szert tettem, illetve több, grafikában alkalmazott praktikát megismertem a számításigény csökkentése érdekében, habár ezek minden részletét már nem fejtettem ki a dolgozatban. 

Megismertem a tűz fizikai és kémiai hátterét, mely segített, hogy a jelenséget kód formájában is meg tudjam fogalmazni. Különféle módszereket mutattam be a megvalósításra vonatkozóan, ezeken elmélkedve és saját ötletekkel kiegészítve több lehetőséget is feltártam az adott módszerekkel kapcsolatban a valószerűség javítása érdekében.
Ezek közül néhányat magam is sikeresen implementáltam, melyet részletesen dokumentáltam a rész- és végeredmények bemutatásával egyetemben. 

%tapasztalatok
% játékmotorokon elmenni nem piskóta, árnyalók végtelen lehetőségei, nehézkes debuggolása, átlátszóság és kirajzolás sorrendje kemény dió, jobb teljesítményhez hardverközelebbi megoldások, ismeretek szükségesek, kreativitás > erőforrások
A dolgozat írása közben azt tapasztaltam, hogy a legapróbb részletekben rejlő hiba is komoly akadályokat jelenthet. Ezen hibák feltárása rengeteg utánajárást és kutatást igényelt, legtöbbször a dokumentációknak, és néha a vak szerencsének köszönhettem a megoldást. A korábbi játékmotorok elemzése során rájöttem, hogy a kreativitás képes felülmúlni a hardver képességeit, azonban ehhez komoly jártasság szükséges az informatikai technológiák területén. 


%fejlesztési lehetőségek
%optimalizálás, tűz elmosás, volumetrikus/voxel tűz, fizikai modellek, tűz terjedés-gyulladás-kioltás, interakció más tárgyakkal, fény és hang jelenségek, parázs darabkák
Számos lehetőség van még az implementációim javítására. A post-processing segítségével például a tűz kinézetén is lehetne javítani, a részecskék életciklusa pedig elbírna egy fizikailag élethűbb modellt. A fény és hangjelenségek csatolása, illetve a más objektumokkal történő interakció megvalósítása szintén segíthetné a hitelesebb megjelenítést. 


\end{document}
