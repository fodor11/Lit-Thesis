\Chapter{CD-melléklet tartalma}

A szakdolgozatom mellé egy darab CD tartozik, amely a következő adatokat tartalmazza:
\begin{itemize}
\item \texttt{dolgozat.pdf}: a dolgozat PDF formátumban.
\item \texttt{Dolgozat} katalógus: a szakdolgozat fájljai.
\item \texttt{SpriteFire} katalógus: a billboard alapú tűz forrásfájljai és egy futtatható állománya.
\item \texttt{ParticleFire} katalógus: a részecskerendszer alapú tűz forrásfájljai és egy futtatható állománya.
\item \texttt{Eredmenyek} katalógus: képernyő videók az eredmények bemutatása céljából.
\item \texttt{osszegzes.pdf}, \texttt{osszegzes.text}: a dolgozat magyar nyelvű összefoglalója.
\item \texttt{summary.pdf}, \texttt{summary.tex}: a dolgozat angol nyelvű összefoglalója.
\end{itemize}

A futtatható fájlok Windows operációs rendszer alatt használhatók.\\
A forráskódokban a tűz implementációk a \texttt{fire.hpp} és \texttt{fire.cpp} fájlokban találhatóak, a kirajzolás a \texttt{main.cpp} fájl \texttt{display} függvényében található. A részecske tűz forrásai közül még a \texttt{postprocessor.hpp} és \texttt{postprocessor.cpp} fájlok lehetnek érdekesek. 

A SpriteFire demóban a + és - gombokkal a tűz sebességét lehet állítani.\\
A ParticleFire demóban az `m' betűvel lehet a szelet be- és kikapcsolni.

%1.	Egy kötelezően Dolgozat nevű katalógus, amely az alábbiakat tartalmazza: 
%      a.	A dolgozatot tartalmazó fájl, a használt szerkesztő formátumában (pl. .docx)
%      b.	A dolgozatot tartalmazó fájl, PDF formátumban, .pdf kiterjesztéssel 
%      c.	A kiírás, külön fájlban. 
%      d.	A dolgozatból kiemelve, egy-egy külön fájlban a magyar és az angol nyelvű összefoglalót, a használt szerkesztő (pl. .docx) formátumában és PDF formátumban is. 

%2.	További katalógusokban minden olyan anyag, ami a dolgozathoz tartozik (forráskódok, működő demó, nagyobb méretű mellékletek stb.). Ez a tartalom természetesen a munka jellegétől függően nagyon sokféle lehet, célszerű megbeszélni a tervezésvezetővel.


