\documentclass[12pt,a4paper]{report}

\usepackage{dolgozat}

\usepackage{hyperref}

\usepackage{listings}
\usepackage{python}
\usepackage{cpp}

\usepackage{wrapfig}

\linespread{1.2}

\begin{document}

\section*{Summary}

The main purpose of the thesis was to explore the implementation possibilities of the fire and the related phenomena in computer graphics. I tried to get an insight into the solutions of the problems related to fire rendering through investigation of available game engines, and examined these methods with respect to realism and computational costs. I dealt with real time rendering in the first place, but also investigated some methods that cannot be computed in real time. I have implemented some of the former ones as well.

I started the investigation with some of the earliest game engines, then moved on to more modern and computationally intense realizations in chronological order. Meanwhile, I also gathered some knowledge in the history of the game engines, and learned about some practices to reduce computational costs of the applications. 

I learned about the physical and chemical background of the fire, which helped me to be able to program the phenomenon. I wrote about various methods for rendering fire and made up some ideas to improve realism. Some of these were also successfully implemented by me, which I documented along with some partial and final results.

While working on the thesis I found, that even the smallest details can cause huge problems. Their solutions required a lot of research, most of the time I was helped out by documentations, and sometimes I solved problems through dumb luck. While analyzing the older game engines I discovered, that you can fight the limitations of the hardware with creativity, although that requires advanced skills.

There are still lots of possibilities for improvement. The appearance of the fire could be enhanced using post-processing, and the particles could use some more realistic physics. Attaching light and sound, along with the possibility to interact with fire would be highly beneficial for realism.

\end{document}
